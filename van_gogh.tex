\documentclass[12pt,letterpaper]{article}
\usepackage[utf8]{inputenc}
\usepackage[T1]{fontenc}
\usepackage{amsmath}
\usepackage{amsfonts}
\usepackage{amssymb}
\usepackage{graphicx}
\usepackage{IEEEtrantools}
\author{Alan Freeman}
\title{GAN Gogh: Use of a GAN to Imitate the Style of Vincent Van Gogh}
\begin{document}
	\maketitle
	\section{Weeks 1-3}
	\subsection{Work Completed}
	During this time period, we tested sample code from the TorchGAN\cite{pal2019torchgan} project using data from the MNIST\cite{lecun2010mnist} dataset. We also began writing code to provide an interface for the TorchGAN code to read from the training data we had previously collected from the Van Gogh Museum (e.g., \cite{001}, \cite{002}, \cite{003}).
	\subsection{Importance of Work}
	Training on the MNIST dataset demonstrated the validity of the training code. By verifying the results in this way, we can state with confidence that any deviation is due to the dataset, rather than the code.
	\subsection{Problems Encountered}
	After working on writing a subclass to use as an interface between the code and the image data, we discovered that the code was redundant, as there was a general-purpose class included in TorchGAN that could be used for that exact purpose. This was a setback, but ultimately allowed us to continue to move forward without having to worry about testing custom code.

	\section{Weeks 4-6}
	\subsection{Work Completed}
	This period of the project was almost entirely spent on training the model on the Van Gogh Museum dataset. We decided that, for the sake of reliability and availability, we would train the model locally.
	\subsection{Importance of Work}
	Training the model represents the bulk of the work of this project. While a certain amount of time has to be allocated for setting things up, the real results of the project are the models produced by training, as well as the images generated by those models.
	\subsection{Problems Encountered}
	Due to the hardware configuration available locally, we were not able to utilize GPU acceleration for our training. This meant slower training, but should not affect the quality of results, as the MNIST test was successful under the same conditions.

	Additionally, despite running the model for 8000 epochs of training, we did not achieve the expected results.
\bstctlcite{IEEEexample:BSTcontrol}
\bibliographystyle{IEEEtran}
\bibliography{van_gogh_bib}
\nocite{*}
\end{document}